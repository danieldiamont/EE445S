%%%%%%%%%%%%%%%%%%%%%%%%%%%%%%%%%%%%%%%%%
% University/School Laboratory Report
% LaTeX Template
% Version 3.1 (25/3/14)
%
% This template has been downloaded from:
% http://www.LaTeXTemplates.com
%
% Original author:
% Linux and Unix Users Group at Virginia Tech Wiki 
% (https://vtluug.org/wiki/Example_LaTeX_chem_lab_report)
%
% License:
% CC BY-NC-SA 3.0 (http://creativecommons.org/licenses/by-nc-sa/3.0/)
%
%%%%%%%%%%%%%%%%%%%%%%%%%%%%%%%%%%%%%%%%%

%----------------------------------------------------------------------------------------
%	PACKAGES AND DOCUMENT CONFIGURATIONS
%----------------------------------------------------------------------------------------

\documentclass{article}

\usepackage{graphicx} % Required for the inclusion of images
\usepackage{amsmath} % Required for some math elements 
\usepackage{cite}
\usepackage{subcaption} %Required to group figures
\usepackage{float}

\setlength\parindent{0pt} % Removes all indentation from paragraphs

%\usepackage{times} % Uncomment to use the Times New Roman font

%----------------------------------------------------------------------------------------
%	DOCUMENT INFORMATION
%----------------------------------------------------------------------------------------

\title{Lab 4\\ Pseudo Random Sequences\\ EE 445S} % Title

\author{Enoc Balderas\\
        \and
        Daniel Diamont\\} % Author name

\date{\today} % Date for the report

\begin{document}

\maketitle % Insert the title, author and date

\begin{center}
\begin{tabular}{l r}
Date Performed: & March 25, 2019 \\ % Date the experiment was performed
Instructor: & Professor Evans % Instructor/supervisor
\end{tabular}
\end{center}

% If you wish to include an abstract, uncomment the lines below
% \begin{abstract}
% Abstract text
% \end{abstract}

%----------------------------------------------------------------------------------------
%	SECTION 1
%----------------------------------------------------------------------------------------

\section{Introduction}

For this lab we explored pseudo random sequences.
The main sequence that we observed is the m-sequence (max length).
We created the m-sequence using a simple shift register (SSRG).
Finally we used the m-sequence at the transmitter to scramble a bit and then used the same sequence at the reciever to descramble the bit.

%----------------------------------------------------------------------------------------
%	SECTION 2
%----------------------------------------------------------------------------------------

\section{Methods}

We started off by creating a $[5,2]_s$ SSRG to implement the sequence.
We tested the sequence by profiling the output with an oscilloscope and making sure that it was periodic.
Next we used the m-sequence to scramble a transmit bit, and subsequently descramble the bit at the reciever.
This was achieved by starting the SSRG of the transmitter and reciever in the same initial state, and xoring the input bit with the resulting sequence.

 
%----------------------------------------------------------------------------------------
%	SECTION 3
%----------------------------------------------------------------------------------------

\section{Results}

\begin{figure}[h]
  \begin{center}
    \includegraphics[width=0.65\textwidth]{img/placeholder.jpg}
    \caption{SSRG $[5,2]_s$.}
  \end{center}
\end{figure}

\textbf{Table:}

\begin{tabular}{c|c}
  count	& state \\ \hline
  [0]	  & 10000 \\ \hline 
  [1]	  & 01000 \\ \hline
  [2]	  & 10100 \\ \hline
  [3]	  & 01010 \\ \hline
  [4]	  & 10101 \\ \hline
  [5]	  & 11010 \\ \hline
  [6]	  & 11101 \\ \hline
  [7]	  & 01110 \\ \hline
  [8]	  & 10111 \\ \hline
  [9]	  & 11011 \\ \hline
  [10]  & 01101 \\ \hline
  [11]	& 00110 \\ \hline
  [12]	& 00011 \\ \hline
  [13]	& 10001 \\ \hline
  [14]	& 11000 \\ \hline
  [15]	& 11100 \\ \hline
  [16]	& 11110 \\ \hline
  [17]	& 11111 \\ \hline
  [18]	& 01111 \\ \hline
  [19]	& 00111 \\ \hline
  [20]	& 10011 \\ \hline
  [21]	& 11001 \\ \hline
  [22]	& 01100 \\ \hline
  [23]	& 10110 \\ \hline
  [24]	& 01011 \\ \hline
  [25]	& 00101 \\ \hline
  [26]	& 10010 \\ \hline
  [27]	& 01001 \\ \hline
  [28]	& 00100 \\ \hline
  [29]	& 00010 \\ \hline
  [30]	& 00001 \\ \hline
  [31]	& 10000 \\ \hline
  [32]	& 01000 \\ \hline
  [33]	& 10100 \\ \hline
  [34]	& 01010 \\ \hline
  [35]	& 10101 \\ \hline
  [36]	& 11010 \\ \hline
  [37]	& 11101 \\ \hline
  [38]	& 01110 \\ \hline
  [39]	& 10111 \\ \hline
  [40]	& 11011 \\ \hline
  [41]	& 01101 \\ \hline
  [42]	& 00110 \\ \hline
  [43]	& 00011 \\ \hline
  [44]	& 10001 \\ \hline
  [45]	& 11000 \\ \hline
  [46]	& 11100 \\ \hline
  [47]	& 11110 \\ \hline
  [48]	& 11111 \\ \hline
  [49]	& 01111 \\ \hline
  [50]	& 00111 \\ \hline
  [51]	& 10011 \\ \hline
  [52]	& 11001 \\ \hline
  [53]	& 01100 \\ \hline
  [54]	& 10110 \\ \hline
  [55]	& 01011 \\ \hline
  [56]	& 00101 \\ \hline
  [57]	& 10010 \\ \hline
  [58]	& 01001 \\ \hline
  [59]	& 00100 \\ \hline
  [60]	& 00010 \\ \hline
  [61]	& 00001 \\ \hline
  [62]	& 10000 \\ \hline
  [63]	& 01000 \\ \hline
  [64]	& 10100 \\ \hline
  [65]	& 01010 \\ \hline
  [66]	& 10101 \\ \hline
  [67]	& 11010 \\ \hline
  [68]	& 11101 \\ \hline
  [69]	& 01110 \\ \hline
  [70]	& 10111 \\ \hline
  [71]	& 11011 \\ \hline
  [72]	& 01101 \\ \hline
  [73]	& 00110 \\ \hline
  [74]	& 00011 \\ \hline
  [75]	& 10001 \\ \hline
  [76]	& 11000 \\ \hline
  [77]	& 11100 \\ \hline
  [78]	& 11110 \\ \hline
  [79]	& 11111 \\ \hline
  [80]	& 01111 \\ \hline
  [81]	& 00111 \\ \hline
  [82]	& 10011 \\ \hline
  [83]	& 11001 \\ \hline
  [84]	& 01100 \\ \hline
  [85]	& 10110 \\ \hline
  [86]	& 01011 \\ \hline
  [87]	& 00101 \\ \hline
  [88]	& 10010 \\ \hline
  [89]	& 01001 \\ \hline
  [90]	& 00100 \\ \hline
  [91]	& 00010 \\ \hline
  [92]	& 00001 \\ \hline
  [93]	& 10000 \\ \hline
  [94]	& 01000 \\ \hline
  [95]	& 10100 \\ \hline
  [96]	& 01010 \\ \hline
  [97]	& 10101 \\ \hline
  [98]	& 11010 \\ \hline
  [99]	& 11101 \\ \hline
\end{tabular}

\begin{figure}[h]
  \begin{center}
    \includegraphics[width=0.65\textwidth]{img/placeholder.jpg}
    \caption{Scrambler and Descrambler.}
  \end{center}
\end{figure}

\textbf{Table:}

\begin{center}
\begin{tabular}{c|c|c}
  count	& scrambler output & descrambler\\ \hline
  [0]	  & 0 & 1 \\ \hline 
  [1]	  & 1 & 1 \\ \hline
  [2]	  & 1 & 1 \\ \hline
  [3]	  & 1 & 1 \\ \hline
  [4]	  & 1 & 1 \\ \hline
  [5]	  & 0 & 1 \\ \hline
  [6]	  & 1 & 1 \\ \hline
  [7]	  & 0 & 1 \\ \hline
  [8]	  & 1 & 1 \\ \hline
  [9]	  & 0 & 1 \\ \hline
  [10]  &	0 & 1 \\ \hline
  [11]	&	0 & 1 \\ \hline
  [12]	&	1 & 1 \\ \hline
  [13]	&	0 & 1 \\ \hline
  [14]	&	0 & 1 \\ \hline
  [15]	&	1 & 1 \\ \hline
  [16]	&	1 & 1 \\ \hline
  [17]	&	1 & 1 \\ \hline
  [18]	&	0 & 1 \\ \hline
  [19]	&	0 & 1 \\ \hline
  [20]	&	0 & 1 \\ \hline
  [21]	&	0 & 1 \\ \hline
  [22]	&	0 & 1 \\ \hline
  [23]	&	1 & 1 \\ \hline
  [24]	&	1 & 1 \\ \hline
  [25]	&	0 & 1 \\ \hline
  [26]	&	0 & 1 \\ \hline
  [27]	&	1 & 1 \\ \hline
  [28]	&	0 & 1 \\ \hline
  [29]	&	1 & 1 \\ \hline
  [30]	&	1 & 1 \\ \hline
  [31]	&	0 & 1 \\ \hline
  [32]	&	1 & 1 \\ \hline
  [33]	&	1 & 1 \\ \hline
  [34]	&	1 & 1 \\ \hline
  [35]	&	1 & 1 \\ \hline
  [36]	&	0 & 1 \\ \hline
  [37]	&	1 & 1 \\ \hline
  [38]	&	0 & 1 \\ \hline
  [39]	&	1 & 1 \\ \hline
  [40]	&	0 & 1 \\ \hline
  [41]	&	0 & 1 \\ \hline
  [42]	&	0 & 1 \\ \hline
  [43]	&	1 & 1 \\ \hline
  [44]	&	0 & 1 \\ \hline
  [45]	&	0 & 1 \\ \hline
  [46]	&	1 & 1 \\ \hline
  [47]	&	1 & 1 \\ \hline
  [48]	&	1 & 1 \\ \hline
  [49]	&	0 & 1 \\ \hline
  [50]	&	0 & 1 \\ \hline
  [51]	&	0 & 1 \\ \hline
  [52]	&	0 & 1 \\ \hline
  [53]	&	0 & 1 \\ \hline
  [54]	&	1 & 1 \\ \hline
  [55]	&	1 & 1 \\ \hline
  [56]	&	0 & 1 \\ \hline
  [57]	&	0 & 1 \\ \hline
  [58]	&	1 & 1 \\ \hline
  [59]	&	0 & 1 \\ \hline
  [60]	&	1 & 1 \\ \hline
  [61]	&	1 & 1 \\ \hline
  [62]	&	0 & 1 \\ \hline
  [63]	&	1 & 1 \\ \hline
  [64]	&	1 & 1 \\ \hline
  [65]	&	1 & 1 \\ \hline
  [66]	&	1 & 1 \\ \hline
  [67]	&	0 & 1 \\ \hline
  [68]	&	1 & 1 \\ \hline
  [69]	&	0 & 1 \\ \hline
  [70]	&	1 & 1 \\ \hline
  [71]	&	0 & 1 \\ \hline
  [72]	&	0 & 1 \\ \hline
  [73]	&	0 & 1 \\ \hline
  [74]	&	1 & 1 \\ \hline
  [75]	&	0 & 1 \\ \hline
  [76]	&	0 & 1 \\ \hline
  [77]	&	1 & 1 \\ \hline
  [78]	&	1 & 1 \\ \hline
  [79]	&	1 & 1 \\ \hline
  [80]	&	0 & 1 \\ \hline
  [81]	&	0 & 1 \\ \hline
  [82]	&	0 & 1 \\ \hline
  [83]	&	0 & 1 \\ \hline
  [84]	&	0 & 1 \\ \hline
  [85]	&	1 & 1 \\ \hline
  [86]	&	1 & 1 \\ \hline
  [87]	&	0 & 1 \\ \hline
  [88]	&	0 & 1 \\ \hline
  [89]	&	1 & 1 \\ \hline
  [90]	&	0 & 1 \\ \hline
  [91]	&	1 & 1 \\ \hline
  [92]	&	1 & 1 \\ \hline
  [93]	&	0 & 1 \\ \hline
  [94]	&	1 & 1 \\ \hline
  [95]	&	1 & 1 \\ \hline
  [96]	&	1 & 1 \\ \hline
  [97]	&	1 & 1 \\ \hline
  [98]	&	0 & 1 \\ \hline
  [99]	&	1 & 1 \\ \hline
\end{tabular}
\end{center}

\textbf{Code:}

\begin{verbatim}
pn = [1, 0, 0, 0, 0, 1, 0, 1, 0, 1, 1, 1, 0, 1, 
      1, 0, 0, 0, 1, 1, 1, 1, 1, 0, 0, 1, 1, 0, 1, 0, 0];

sc = [0, 1, 1, 1, 1, 0, 1, 0, 1, 0, 0, 0, 1, 0, 
      0, 1, 1, 1, 0, 0, 0, 0, 0, 1, 1, 0, 0, 1, 0, 1, 1];

pn(pn == 0) = -1;
sc(sc == 0) = -1;

tmp1 = [pn pn];
tmp2 = [sc sc];

s1 = fft(tmp1);
pn_corr = ifft(s1.*conj(s1))/length(tmp1);

s2 = fft(tmp2);
sc_corr = ifft(s2.*conj(s2)/length(tmp2));
\end{verbatim}

\begin{figure}[h]
  \begin{center}
    \includegraphics[width=0.65\textwidth]{img/pn_corr.png}
    \caption{pseudo random sequence autocorrelation.}
  \end{center}
\end{figure}

\begin{figure}[h]
  \begin{center}
    \includegraphics[width=0.65\textwidth]{img/sc_corr.png}
    \caption{scrambled bit autocorrelation.}
  \end{center}
\end{figure}


%----------------------------------------------------------------------------------------
%	SECTION 4
%----------------------------------------------------------------------------------------

\section{Discussion}


%----------------------------------------------------------------------------------------
%	BIBLIOGRAPHY
%----------------------------------------------------------------------------------------

\bibliography{mybib}
\bibliographystyle{plain}

%----------------------------------------------------------------------------------------


\end{document}
